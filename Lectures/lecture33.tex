\ProvidesFile{lecture33.tex}[Лекция 33]


В случае евклидова пространства нам нужно одно техническое наблюдение.

\begin{claim}
\label{claim::OrthoRealfix}
Пусть $V$ -- евклидово пространство, $V_\mathbb C$ -- его комплексификация и $w\in V_\mathbb C$.
Тогда эквивалентно
\begin{enumerate}
\item $w\bot \bar w$

\item выполнены два условия:
\begin{enumerate}
\item $|\Re w | = |\Im w|$

\item $\Re w \bot \Im w$
\end{enumerate}
\end{enumerate}
\end{claim}
\begin{proof}
Пусть $w = v+ iu$.
Тогда, расписав по определению, получим $(w, \bar w) = (v, v) - (u,u) +2i(v,u)$.
Значит, 
\[
w\bot \bar w \iff
\left\{
\begin{aligned}
&(v,v) - (u,u) = 0\\
&(v,u) = 0
\end{aligned}
\right.
\iff
\left\{
\begin{aligned}
|v| &= |u|\\
v &\bot u
\end{aligned}
\right.
\]
\end{proof}

\begin{claim}
\label{claim::OrthotInvarSub}
Пусть $V$ евклидово пространство и $\phi \colon V\to V$ -- некоторое движение.
Тогда
\begin{itemize}
\item Либо существует собственный вектор $v\in V$ для собственных значений $\pm 1$.

\item Либо существует инвариантное двумерное подпространство $U\subseteq V$ и его ортонормированный базис $e_1, e_2$ такие, что матрица $\phi|_U$ в этом базисе имеет вид
\[
\begin{pmatrix}
{\cos \alpha}&{-\sin \alpha}\\
{\sin \alpha}&{\cos \alpha}
\end{pmatrix}
,\;\text{для некоторого $\alpha \in [0,2\pi)$}
\]
\end{itemize}
\end{claim}
\begin{proof}
По утверждению~\ref{claim::MovmentBasicProp} все вещественные собственные значения $\phi$ -- это $\pm 1$.
Значит, если спектр не пуст, то выполнено первое условие.
Пусть $\spec_\mathbb R= \varnothing$.
В этом случае перейдем к комплексификации отображения $\phi_\mathbb C\colon V_\mathbb C\to V_\mathbb C$.
По утверждению~\ref{claim::MovementComplfix} $\phi_\mathbb C$ является движением в эрмитовом пространстве.
Так как $\mathbb C$ алгебраически замкнуто, то спектр обязательно не пуст.
А по утверждению~\ref{claim::MovmentBasicProp} в нем обязательно числа по модулю равные $1$.
Значит найдется $\lambda = \cos \alpha - i \sin \alpha$ и $w= v + i u\in V_\mathbb C$ такие, что $\phi_\mathbb C(w) = \lambda w$.
Давайте распишем эти условия более явно.
С одной стороны по определению:
\[
\phi_\mathbb C(w) = \phi(v) + i\phi(u)
\]
С другой стороны, так как $w$ собственные, имеем
\[
\phi_\mathbb C(w) = \lambda w = (\cos \alpha - i \sin \alpha) (v+ i u) = \cos \alpha \;v + \sin \alpha\; u + (-\sin \alpha\; v +\cos\alpha\; u)
\]
Но в комплексификации вектора равны тогда и только тогда, когда их вещественные и мнимые части равны.
Значит,
\begin{align*}
\phi(v) &= \cos \alpha\; v + \sin \alpha\; u\\
\phi(u) &= -\sin \alpha\; v +\cos\alpha\; u
\end{align*}
Таким образом подпространство в $V$ натянутое на векторы $v$ и $u$ является $\phi$ инвариантым.
Если я теперь покажу, что векторы $v$ и $u$ ортогональны и имеют одинаковую длину, то пространство $\langle v,u\rangle$ будет двумерным и, поделив их на их общую длину, я получу ортонормированный базис и матрица оператора в этом базисе будет ровно такой, как и заявлено.

Давайте применим комплексное сопряжение к равенству $\phi_\mathbb C(w) = \lambda w$.
Тогда левая часть
\[
\overline{\phi_\mathbb C(w)} = \overline{\phi(v) + i \phi(w)} = \phi(v) - i \phi(w) = \phi_\mathbb C(v - iu) = \phi_\mathbb C(\overline{v + i w}) = \phi_\mathbb C(\bar w)
\]
А правая часть равно
\[
\overline{\lambda w} = \bar \lambda \bar w
\]
То есть $\bar w$ -- это собственный вектор с собственным значением $\bar \lambda$.
Но так как $\lambda$ было не вещественным числом, $\lambda \neq \bar\lambda$.
А значит, по утверждению~\ref{claim::MovmentBasicProp} векторы $w$ и $\bar w$ ортогональны.
Но тогда по утверждению~\ref{claim::OrthoRealfix} $|v| = |u|$ и $(v, u) = 0$.
При этом длина этих векторов не ноль, иначе $w$ был бы нулем, что противоречило бы его выбору.

\end{proof}

\begin{claim}
Пусть $V$ евклидово пространство и $\phi\colon V\to V$ -- некоторый оператор.
Тогда эквивалентно
\begin{enumerate}
\item $\phi$ является движением (ортогональный оператор).

\item В некотором ортонормированном базисе матрица оператора $\phi$ имеет вид:
\[
A_\phi=
\begin{pmatrix}
{A_1}&{}&{}\\
{}&{\ddots}&{}\\
{}&{}&{A_r}\\
\end{pmatrix},
\quad\text{где}\quad
A_i\;\;\text{либо}\;\;1,\;\;\text{либо}\;\;-1,\;\;\text{либо}\;\;
\begin{pmatrix}
{\cos \alpha}&{-\sin\alpha}\\
{\sin\alpha}&{\cos\alpha}
\end{pmatrix}
\]
\end{enumerate}
\end{claim}
\begin{proof}
(2)$\Rightarrow$(1).
Пусть $e_1,\ldots,e_n$ -- тот самый ортонормированный базис и $A$ -- матрица оператора $\phi$.
Прямая проверка показывает, что $A^tA = E$.
Тогда по утверждению~\ref{claim::MovementMatrix} (описание движений в терминах матриц) $\phi$ является движением.

(1)$\Rightarrow$(2).
По утверждению~\ref{claim::OrthotInvarSub} возможен один из двух случаев:
\begin{enumerate}
\item Существует собственный вектор $v$ с собственным значением $1$ или $-1$.

\item Существует инвариантное двумерное подпространство $U$ с ортонормированным базисом $e_1, e_2$ такие, что матрица $\phi|U$ в этом базисе имеет указанный вид.
\end{enumerate}

Если у нас случай (1), то мы рассмотрим разложение пространства $V = \langle v\rangle \oplus \langle v\rangle^\bot$.
Тогда ортогональное дополнение будет $\phi$-инвариантным.
Значит мы можем ограничить на него $\phi$, по индукции там найти ортонормированный базис $e_2,\ldots,e_n$ с требуемыми свойствами.
Тогда положим $e_1 = v / |v|$ и базис $e_1,\ldots,e_n$ будет искомым.

Если у нас случай (2), то мы рассмотрим разложение $V = U \oplus U^\bot$.
Тогда ортогональное дополнение будет $\phi$-инвариантным.
Значит мы можем ограничить на него $\phi$ и как и раньше найти ортонормированный базис $e_3,\ldots,e_n$ с требуемыми свойствами.
Тогда возьмем $e_1,e_2$ из $U$ и базис $e_1,\ldots,e_n$ будет искомым.
\end{proof}

\begin{claim}
Пусть $V$ -- вещественное векторное пространство и $\phi\colon V\to V$ -- некоторый оператор.
Тогда эквивалентно:
\begin{enumerate}
\item Существует скалярное произведение на $V$, относительно которого $\phi$ является движением.

\item В некотором базисе матрица оператора $\phi$ имеет вид:
\[
A_\phi=
\begin{pmatrix}
{A_1}&{}&{}\\
{}&{\ddots}&{}\\
{}&{}&{A_r}\\
\end{pmatrix},
\quad\text{где}\quad
A_i\;\;\text{либо}\;\;1,\;\;\text{либо}\;\;-1,\;\;\text{либо}\;\;
\begin{pmatrix}
{\cos \alpha}&{-\sin\alpha}\\
{\sin\alpha}&{\cos\alpha}
\end{pmatrix}
\]
\end{enumerate}
\end{claim}
\begin{proof}
(1)$\Rightarrow$(2).
Если существует скалярное произведение, то оно задает структуру евклидова пространства, а значит мы можем применить предыдущее утверждение.

(2)$\Rightarrow$(1).
Пусть $e_1,\ldots,e_n$ -- базис, в котором $\phi$ имеет указанный блочный вид.
Зададим скалярное произведение так, чтобы $e_1,\ldots,e_n$ был ортонормированным (так сделать можно по утверждению~\ref{claim::ScalarDef}).
Тогда выполнен пункт~(2) предыдущего утверждения, а значит и выполнен пункт~(1).
То есть $\phi$ является движением относительно построенного скалярного произведения.
\end{proof}


\subsection{Сопряженное линейное отображение}

В случае произвольного линейного оператора $\phi$ на пространстве $V$, его сопряженный или двойственный $\phi^*$ живет на $V^*$ и это неудобно.
В случае евклидова или эрмитова пространства можно определить сопряженный оператор уже на самом пространстве $V$.
Я в начале расскажу, как строится подобный оператор на пространстве $V$ с помощью скалярного произведения, а потом уже поясню, как он связан с нашим старым знакомым на двойственном пространстве.

\begin{claim}
Пусть $V$ и $U$ -- евклидовы или эрмитовы пространства и $\phi\colon V\to U$ линейное отображение.
Тогда существует единственное линейное отображение $\phi^*\colon U\to V$ такое, что $(\phi v, u) = (v, \phi^* u)$ для любых $v\in V$ и $u\in U$.
\end{claim}
\begin{proof}
Для доказательства перейдем в базисы и сделаем все на матричном языке.
Давайте я для определенности рассмотрю эрмитов случай.
Пусть $e_1,\ldots,e_n$ ортонормированный базис пространства $V$, а $f_1,\ldots,f_m$ -- ортонормированный базис пространства $U$.
Тогда $V$ превращается в $\mathbb C^n$, $U$ превращается в $\mathbb C^m$, скалярное произведение в обоих пространствах становится стандартным, а отображение $\phi$ превращается в отображение $\mathbb C^n \to \mathbb C^m$ по правилу $x \mapsto Ax$ для некоторой $A\in\operatorname{M}_{m\,n}(\mathbb C)$.
Давайте будем искать наше отображение $\phi^*$  в виде $\phi^*(x) = Bx$, где $B\in \operatorname{M}_{n\,m}(\mathbb C)$ и покажем, что существует ровно одно линейное отображение удовлетворяющее нужным условиям.

Действительно, условие $(\phi v, u) = (v, \phi^* u)$ переписывается так
\[
(\overline{Ax})^t y = \bar x^t By\iff \bar x^t \bar A^t y = \bar x^t B y
\]
И это должно выполняться для любых $x\in \mathbb C^n$ и $y\in \mathbb C^m$.
А это возможно тогда и только тогда, когда $B = \bar A^t$.
Что и означает, что нужное отображение найдется и единственное.
\end{proof}

\begin{definition}
Если $V$ и $U$ -- евклидовы или эрмитовы пространства и $\phi\colon V\to U$ -- линейное отображение.
Тогда отображение $\phi^*\colon U\to V$ называется сопряженным к $\phi$.
В частности, если $\varphi\colon V\to V$ является линейным оператором, то $\varphi^*$ называется сопряженным оператором.%
\footnote{Иногда говорят евклидово сопряженный или эрмитово сопряженный, чтобы подчеркнуть, что мы в евклидовом или эрмитовом случае.}
\end{definition}

\paragraph{Замечания}

\begin{itemize}
\item В силу симметричности скалярного произведения определение сопряженного отображения можно дать в виде $(\phi^* u, v) = (u, \phi v)$.
Таким образом по простому, сопряженное линейное отображение  -- это такое линейное отображение, которое получается при перекидывании внутри скалярного произведения.
Это означает, что если вы хотите что-то доказать для сопряженного линейного отображения, то надо желаемый факт выразить в терминах скалярного произведения, а после этого перекинуть $\phi^*$ и превратить его в $\phi$ и воспользоваться свойствами $\phi$ или наоборот.

\item Обратите внимание, что если у нас оператор $\phi\colon V\to V$ в ортонормированном базисе задан матрицей $A$, то оператор $\phi^*$ в этом же базисе будет задан матрицей $A^t$ в вещественном случае и $A^* = \bar A^t$ в комплексном случае.

\item Пусть теперь $\phi\colon V\to U$ -- произвольное линейное отображение.
Пусть $e_1,\ldots,e_n$ -- некоторый базис в $V$ и $f_1,\ldots, f_m$ -- некоторый базис в $U$.
Пусть скалярное произведение $V$ в указанном базисе задано матрицей $B$, а в пространстве $U$ в указанном базисе задано матрицей $G$.
Тогда если $\phi$ задан матрицей $A$, то матрица сопряженного линейного отображения будет $B^{-1} A^t G$ в вещественном случае и $B^{-1}\bar A^t G$ в комплексном.

\item Если $\phi\colon V\to V$ движение, то это значит, что $(\phi v, \phi u) = (v, u)$.
Если обозначить $\phi u $ за $w$, то получим, что $\phi$ движение, тогда и только тогда, когда $(\phi v, w) = (v, \phi^{-1}w)$ для любых $v,w\in V$.
То есть $\phi$ движение, тогда и только тогда, когда $\phi^* = \phi^{-1}$.
Таким образом движения можно выразить в терминах сопряженного оператора.

\item В общем случае даже не пытайтесь понять геометрический смысл сопряженного оператора.
Это совсем не очевидная штука.
\end{itemize}

\paragraph{Связь с двойственным пространством}

Здесь мне придется разобрать отдельно евклидов и эрмитов случай.
Пусть $V$ и $U$ -- евклидовы пространства и $\phi\colon V\to U$ -- линейное отображение.
Тогда существует двойственное линейное отображение $\phi^*\colon U^*\to V^*$ по правилу $\xi \mapsto \xi \phi$.
Кроме этого, скалярное произведение индуцирует изоморфизм $V\to V^*$ по правилу $v\mapsto ({-},v)$ и  аналогично для $U$.
В итоге получаем следующую диаграмму
\[
\xymatrix{
	{V^*}&{U^*}\ar[l]_{\phi^*}\\
	{V}\ar@{~>}[u]&{U}\ar@{~>}[u]\ar@{-->}[l]_{\psi}
}
\quad
\xymatrix{
	{({-},\psi u) = (\phi({-}), u)}&{({-}, u)}\ar@{|->}[l]\\
	{\psi u}\ar@{|->}[u]&{u}\ar@{|->}[u]\ar@{|->}[l]
}
\]
Вертикальные стрелки -- это изоморфизмы с помощью скалярного произведения, а нижняя пунктирная стрелка -- это композиция: сначала изоморфизм $U\to U^*$, потом $\phi^*$, потом обратный изоморфизм $V^* \to V$.
Давайте посмотрим, что получается в качестве отображения $\psi$.
Для этого возьмем произвольный вектор $u\in U$ и пройдем двумя путями в $V^*$.
Справа показан расчет результатов.
Получаем, что функции $({-},\psi u)$ и $(\phi({-}), u)$ на $V$ совпадают при любых $u\in U$, то есть для любого $v\in V$ и любого $u\in U$ имеем равенство $(v, \psi u) = (\phi v, u)$.
Теперь мы видим, что $\psi$ совпадает с определением сопряженного линейного отображения $\phi^*\colon U\to V$ .


Теперь давайте разберемся с комплексным случаем.
Ситуация здесь похожая, но появляется одна тонкость.
Но обо всем по порядку.
Пусть $V$ и $U$ -- эрмитовы пространства и $\phi\colon V\to U$ -- линейное отображение.
Однако, в этом случае отображение $V\to V^*$ по правилу $v\mapsto (v, {-})$ не является $\mathbb C$-линейным, а отображение $v\mapsto ({-}, v)$ не корректно, потому что результат не является $\mathbb C$-линейной функцией на $V$.
Это правится переходом к пространствам $\bar V$ и $\bar U$.
А именно, отображение $\phi\colon V\to U$ определяет отображение $\bar \phi\colon \bar V\to \bar U$ по правилу $v\mapsto \phi(v)$, то есть мы действуем так же, как и исходное отображение $\phi$.
Кроме этого отображение $V\to \bar V^*$ по правилу $({-}, v)$ является $\mathbb C$-линейным изоморфизмом.
Теперь имеем следующую диаграмму в комплексном случае
\[
\xymatrix{
	{\bar V^*}&{\bar U^*}\ar[l]_{\bar \phi^*}\\
	{V}\ar@{~>}[u]&{U}\ar@{~>}[u]\ar@{-->}[l]_{\psi}
}
\quad
\xymatrix{
	{({-},\psi u) = (\phi({-}), u)}&{({-}, u)}\ar@{|->}[l]\\
	{\psi u}\ar@{|->}[u]&{u}\ar@{|->}[u]\ar@{|->}[l]
}
\]
И как и в вещественном случае раскручивание определений приводит нас к равенству $(v, \psi u) = (\phi v, u)$ для всех $v\in V$ и $u\in U$, что означает, что $\psi$ совпадает с определением сопряженного оператора.

\paragraph{Альтернатива Фредгольма}

В случае эрмитовых или евклидовых пространств можно так же сформулировать связь между ядрами и образами оператора и сопряженного к нему.
Получаем следующее утверждение.

\begin{claim}
[Альтернатива Фредгольма]
\label{claim::EuclidHermitFredholm}
Пусть $V$ -- евклидово или эрмитово пространство и $\varphi \colon V\to V$ -- некоторый линейный оператор и $\varphi^*\colon V\to V$ -- его сопряженный.
Тогда
\begin{enumerate}
\item $\ker\varphi^* = \Im\varphi^\perp$.

\item $\Im \varphi^* = \ker \varphi^\perp$.
\end{enumerate}
\end{claim}
\begin{proof}
(1) По определению
\[
\Im \varphi ^\perp = \{v\in V \mid v \perp \Im\varphi\} = \{v\in V\mid (v, \varphi(u)) = 0,\;\forall\,u\in V\} = \{v\in V\mid (\varphi^*(v), u) = 0,\;\forall\, u\in V\}
\]
Последнее условие означает, что $\varphi^*(v)$ лежит в ядре скалярного произведения, которое равно нулю по определению.
Значит последнее пространство совпадает с
\[
\{v\in V\mid \varphi^*(v)\} = \ker \varphi^*
\]
Что и требовалось.

(2) Этот пункт следует из первого.
Для этого обозначим $\psi = \varphi^*$.
Тогда $\varphi = \psi^*$.
Теперь нам надо показать, что $\Im \psi = (\ker \psi^*)^\perp$.
Но в силу двойственности для подпространств (см.~утверждение~\ref{claim::DualitySpaces}) это равносильно условию $\Im \psi^\perp = \ker \psi^*$.
А это как раз первый пункт примененные к оператору $\psi$ вместо $\varphi$.	
\end{proof}

\begin{definition}
Пусть $V$ -- евклидово или эрмитово пространство и $\varphi \colon V\to V$ -- линейный оператор.
Оператор $\varphi$ называется нормальным, если $\varphi \varphi^* = \varphi^* \varphi$.
\end{definition}

\paragraph{Замечания}

\begin{itemize}
\item Мы уже отмечали, что оператор $\varphi$ является движением тогда и только тогда, когда $\varphi^* = \varphi^{-1}$.
Кроме того, другой интересный класс операторов, которому будет посвящен весь следующий раздел -- это так называемые самосопряженные операторы с условием $\varphi^* = \varphi$.
В обоих случаях выполняется условие $\varphi \varphi^* = \varphi^*\varphi$.
Таким образом это все частные случаи нормальных операторов.

\item В эрмитовом случае теорию движений и самосопряженных операторов можно выводить из общей теории нормальных операторов.
Однако, я предпочитаю доказать параллельно ключевые факты про каждый из этих классов операторов.
\end{itemize}
